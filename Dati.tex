%----------------------------------------------------------------------------------------
%   USEFUL COMMANDS
%----------------------------------------------------------------------------------------

\newcommand{\dipartimento}{Dipartimento di Matematica ``Tullio Levi-Civita''}

%----------------------------------------------------------------------------------------
% 	USER DATA
%----------------------------------------------------------------------------------------

% Data di approvazione del piano da parte del tutor interno; nel formato GG Mese AAAA
% compilare inserendo al posto di GG 2 cifre per il giorno, e al posto di 
% AAAA 4 cifre per l'anno
\newcommand{\dataApprovazione}{Data}

% Dati dello Studente
\newcommand{\nomeStudente}{Tommaso}
\newcommand{\cognomeStudente}{Di Fant}
\newcommand{\matricolaStudente}{1217932}
\newcommand{\emailStudente}{tommaso.difant@gmail.com}
\newcommand{\telStudente}{+ 39 3425706474}

% Dati del Tutor Aziendale
\newcommand{\nomeTutorAziendale}{Lorenzo}
\newcommand{\cognomeTutorAziendale}{Braghetto}
\newcommand{\emailTutorAziendale}{lorenzo@team.unipiazza.it}
\newcommand{\telTutorAziendale}{+ 39 333 6468264}
\newcommand{\ruoloTutorAziendale}{CTO & co-founder}

% Dati dell'Azienda
\newcommand{\ragioneSocAzienda}{Unipiazza S.R.L.}
\newcommand{\indirizzoAzienda}{Via Savonarola 115}
\newcommand{\sitoAzienda}{www.unipiazza.it}

% Dati del Tutor Interno (Docente)
\newcommand{\titoloTutorInterno}{Prof.}
\newcommand{\nomeTutorInterno}{NomeDocente}
\newcommand{\cognomeTutorInterno}{CognomeDocente}

\newcommand{\prospettoSettimanale}{
     % Personalizzare indicando in lista, i vari task settimana per settimana
    \begin{itemize}
        \item \textbf{Prima settimana}
        \begin{itemize}
            \item Introduzione a Unipiazza, il servizio fedeltà, il suo funzionamento e la struttura organizzativa interna;
            \item Presa visione degli strumenti utilizzati per il telelavoro;
            \item Panoramica dei prodotti tecnologici creati dal team Unipiazza;
            \item Presa visione e formazione sulle tecnologie coinvolte.;
            \item Configurazione dell'ambiente di lavoro;
        \end{itemize}
        \item \textbf{Seconda settimana} 
        \begin{itemize}
            \item Configurazione e prima esecuzione progetti web in AngularJs, React e backend in Nodejs;
            \item Studio e presa visione del sistema di rilascio e sviluppo dell'ambiente production e staging;
            \item Esercizio di riconoscimento ed eliminazione di utenti "fake" attualmente nel database. Scelta di come si possano eliminare (task nodejs o a mano dal frontend);
        \end{itemize}
        \item \textbf{Terza settimana} 
        \begin{itemize}
            \item Studio delle best practice sulla scrittura di documentazione e sviluppo API stesse;
            \item Focus sul rendere le API pubbliche sicure dal punto di vista di autenticazione;
        \end{itemize}
        \item \textbf{Quarta settimana} 
        \begin{itemize}
            \item Scelta dei tool da usare per la creazione della documentazione;
            \item Sviluppo e documentazione della parte di autenticazione delle API pubbliche;
        \end{itemize}
        \item \textbf{Quinta e sesta settimana} 
        \begin{itemize}
            \item Documentazione sul resto delle API;
            \item Sincronizzazione con la parte "marketers" del team in modo da dare valore alle API da un punto di vista commerciale e ricerca investors.;
        \end{itemize}
        \item \textbf{Settima e ottava e nona settimana} 
        \begin{itemize}
            \item Studio delle best practice sulla scrittura di test automatici per il backend in nodejs;
            \item Scelta del tool per l'esecuzione dei test;
            \item Scrittura test automatici;
            \item Condivisione risultati, stress test;
        \end{itemize}
        \item \textbf{Decima settimana} 
        \begin{itemize}
            \item Studio e ricerca di possibili oppurtunità che la nuova frontiera dell'Intelligenza Artificiale può portare ad Unipiazza;
            \item Studio della gran mole di dati che Unipiazza possiede;
        \end{itemize}
        \item \textbf{Undicesima e dodicesima settimana - Conclusione} 
        \begin{itemize}
            \item Recupero del lavoro precedentemente condotto dall'ultimo tirocinante relativo l'analisi del comportamento dei nostri utenti. Capire se la AI può aiutare a sviluppare alcune funzionalità.
            \item A discrezione dello studente chiederemo qualche task di proprio gradimento che a suo parere possano essere utili al team;
            \item Inserimento di un proprio easter egg nascosto in una delle varie applicazioni Unipiazza;
        \end{itemize}
    \end{itemize}
}

% Indicare il totale complessivo (deve essere compreso tra le 300 e le 320 ore)
\newcommand{\totaleOre}{300}

\newcommand{\obiettiviObbligatori}{
    \item \underline{\textit{O01}}: Capacità di organizzazione e lavoro in gruppo, destrezza nell'utilizzo degli strumenti già utilizzati dal team (Github, Trello, Slack/..);
    \item \underline{\textit{O02}}: Presa coscienza della complessità di un progetto tecnologicamente variegato e in produzione da ormai 6 anni;
    \item \underline{\textit{O03}}: Raggiungere una buona capacità a integrarsi con codice già scritto da altri; 
    \item \underline{\textit{O04}}: Perfezionamento delle skill di DevOps realtive a scrittura di test automatici e documentazione;
    \item \underline{\textit{O05}}: Ricerca sulle attuali opportunità che Intelligenza Artificiale e Machine Learning posso portare alla gestione e analisi di database di una certa grandezza;
}

\newcommand{\obiettiviDesiderabili}{
    \item \underline{\textit{D01}}: Soluzioni applicabili per poter sfruttare in maniera efficace i dati disponibili da Unipiazza;
}

\newcommand{\obiettiviFacoltativi}{
    \item \underline{\textit{F01}}: Sviluppo effettivo delle soluzioni di cui sopra;
}