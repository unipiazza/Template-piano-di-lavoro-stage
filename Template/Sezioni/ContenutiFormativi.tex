\section*{Contenuti formativi previsti}
% Personalizzare indicando le tecnologie e gli ambiti di interesse dello stage
Per la natura diversificata dei prodotti Unipiazza lo stage prevede l'inserimento dello studente in diverse aree tematiche tecnologiche.
Durante la prima fase lo Stagista prenderà confidenza con l'interfaccia API utilizzata dal nostro server NodeJs, imposterà ed eseguirà i frontend in React e AngularJS e studierà la loro integrazione.\\

Attualmente l'approccio ai test in Unipiazza è strutturato, ma completamente manuale. Tutor e Stagista lavoreranno assieme per implementare finalmente i primi test automatici concentrando il focus soprattutto sul backend in NodeJs.\\

Una parte importante del prodotto Unipiazza vive grazie ai dati raccolti sul comportamento degli utenti finali; mentre da un punto di vista strettamente tecnico le API sono ad un ottimo livello di sviluppo, la documentazione e l'apertura verso qualcunque realtà si volesse integrare con noi è molto sono molto limitate.
Strutturare una piattaforma di "Open API" è un ottimo esempio di procedure DevOps che coinvolgono aspetti di sviluppo, documentazione e comunicazione al tempo stesso.
Nella fase finale dello stage sarà interessante applicare il know how acquisito per ragionare e ricercare possibili nuove soluzioni per l'utilizzo dei dati attraverso algoritmi che li sfruttino.\vspace{5mm}
\newpage